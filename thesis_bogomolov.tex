%% Простая презентация с примером включения программного кода и
%% пошаговых спецэффектов
\documentclass[xcolor=table]{beamer}
\usetheme{SPbAU}
%\useoutertheme{infolines}
\usepackage{fontspec}
\usepackage{xunicode}
\usepackage{xltxtra}
\usepackage{xecyr}
\usepackage{hyperref}
\setmainfont[Mapping=tex-text]{DejaVu Serif}
\setsansfont[Mapping=tex-text]{DejaVu Sans}
\setmonofont[Mapping=tex-text]{DejaVu Sans Mono}
\usepackage{polyglossia}
\setdefaultlanguage{english}
\usepackage{graphicx}
\usepackage[normalem]{ulem} 
\usepackage{listings}
\lstdefinestyle{mycode}{
  belowcaptionskip=1\baselineskip,
  breaklines=true, 
  xleftmargin=\parindent,
  showstringspaces=false,
  basicstyle=\footnotesize\ttfamily,
  keywordstyle=\bfseries,
  commentstyle=\itshape\color{gray!40!black},
  stringstyle=\color{red},
  numbers=left,
  numbersep=5pt,
  numberstyle=\tiny\color{gray},
}
\lstset{escapechar=@,style=mycode}
\lstdefinelanguage{Kotlin}{
  comment=[l]{//},
  commentstyle={\color{gray}\ttfamily},
  emph={delegate, filter, first, firstOrNull, forEach, lazy, map, mapNotNull, println, return@},
  emphstyle={\color{OrangeRed}},
  identifierstyle=\color{black},
  keywords={abstract, actual, as, as?, break, by, class, companion, continue, data, do, dynamic, else, enum, expect, false, final, for, fun, get, if, import, in, interface, internal, is, null, object, override, package, private, public, return, set, super, suspend, this, throw, true, try, typealias, val, var, vararg, when, where, while},
  keywordstyle={\color{NavyBlue}\bfseries},
  morecomment=[s]{/*}{*/},
  morestring=[b]",
  morestring=[s]{"""*}{*"""},
  ndkeywords={@Deprecated, @JvmField, @JvmName, @JvmOverloads, @JvmStatic, @JvmSynthetic, Array, Byte, Double, Float, Int, Integer, Iterable, Long, Runnable, Short, String},
  ndkeywordstyle={\color{BurntOrange}\bfseries},
  sensitive=true,
  stringstyle={\color{ForestGreen}\ttfamily},
}
\hypersetup{
    bookmarks=true,         % show bookmarks bar?
    unicode=false,          % non-Latin characters in Acrobat’s bookmarks
    pdftoolbar=true,        % show Acrobat’s toolbar?
    pdfmenubar=true,        % show Acrobat’s menu?
    pdffitwindow=false,     % window fit to page when opened
    pdfstartview={FitH},    % fits the width of the page to the window
    pdftitle={My title},    % title
    pdfauthor={Author},     % author
    pdfsubject={Subject},   % subject of the document
    pdfcreator={Creator},   % creator of the document
    pdfproducer={Producer}, % producer of the document
    pdfkeywords={keyword1, key2, key3}, % list of keywords
    pdfnewwindow=true,      % links in new PDF window
    colorlinks=false,       % false: boxed links; true: colored links
    linkcolor=red,          % color of internal links (change box color with linkbordercolor)
    citecolor=green,        % color of links to bibliography
    filecolor=magenta,      % color of file links
    urlcolor=cyan           % color of external links
}


\usepackage{array}
\newcolumntype{L}[1]{>{\raggedright\let\newline\\\arraybackslash\hspace{0pt}}m{#1}}
\newcolumntype{C}[1]{>{\centering\let\newline\\\arraybackslash\hspace{0pt}}m{#1}}
\newcolumntype{R}[1]{>{\raggedleft\let\newline\\\arraybackslash\hspace{0pt}}m{#1}}

\usepackage[style=verbose,backend=biber]{biblatex}
\addbibresource{\jobname.bib}

% To change the total number of slides go to beamerouterThemeSPbAU.sty -> line 93 -> change 15 to whatever you want
% If you can do it in some cleaner way without breaking enumeration of slides after "results", go ahead and share your work :)

\begin{document}

\title[Определение авторства разработчиков]{Определение авторства разработчиков на основании стиля написания кода}
\author[Богомолов Е. О.]{Богомолов Е. О.\\~\\{\footnotesize\textcolor{gray}{Научный руководитель: к.ф.-м.н. Д. Ю. Булычев\\Научный консультант: к.т.н. Т. А. Брыксин}}}
\institute{Санкт-Петербургская школа физико-математических и компьютерных наук\\~\\НИУ ВШЭ -- Санкт-Петербург}
\frame{\titlepage}

\setbeamertemplate{frametitle}[default][center]

\begin{frame}
	\frametitle{Задача определения авторства}
	\begin{center}
		\includegraphics[width=0.9\textwidth]{images/authorship-rus.pdf}
	\end{center}
\end{frame}

\begin{frame}
	\frametitle{Мотивация}
	\Large
	\begin{itemize}
		\item Поиск авторов вредоносного ПО
		\item Поиск плагиата
		\item Проверка авторства
		\item Профилирование программистов
	\end{itemize}
\end{frame}

\begin{frame}
	\frametitle{Обзор работ}
	% \setlength\itemsep{1em}
	C++\footcite{caliskan2015}
	\begin{itemize}
		\item Используются лексические и синтаксические факторы, форматирование
		\item Обучается случайный лес из 300 деревьев
		\item Тестирование на решениях 9 задач Google Code Jam, 1600 участников
	\end{itemize}

	Python\footcite{alsulami2017}
	\begin{itemize}
		\item Обучается tree-LSTM по AST и токенам
		\item Тестирование на решениях 10 задач Google Code Jam, 70 участников
	\end{itemize}
\end{frame}

\begin{frame}
	\frametitle{Обзор работ}
	Java\footcite{yang2017}
	\begin{itemize}
		\item Используются лексические, семантические факторы, форматирование
		\item Обучается модель из трех полносвязных слоев при помощи:
		\begin{itemize}
			\item Стохастического градиентного спуска
			\item Метода роя частиц\footcite{Kennedy1995}
		\end{itemize}
		\item Тестирование на 40 проектах с открытым исходным кодом, у каждого один автор
		\item Всего 3021 файл, в проектах от 11 до 712		
	\end{itemize}
\end{frame}


\begin{frame}
	\frametitle{Особенности существующих решений}
	\begin{itemize}
		\item Данные существенно отличаются от промышленного кода
		\begin{itemize}
			\item Специфичные источники данных: соревнования, примеры из книг, домашние задания, проекты с одним автором
			\item Число фрагментов доступных для автора в пределах от 9 до 800
		\end{itemize}
		\item Для различных языков разные решения показывают лучшие результаты
		\item Лучшие решения не адаптируются для произвольного языка
	\end{itemize}
\end{frame}

\begin{frame}
	\frametitle{Цель и задачи}
	\textbf{Цель:} создать модель для определения авторства по коду, работающую с произвольными языками программирования и объёмами данных\\~\\
	\textbf{Задачи:}
	\begin{itemize}
		\item Создать инструмент для сбора примеров кода отдельных авторов из произвольных проектов
		\item Собрать датасеты для тестирования моделей в разных условиях
		\item Создать прототипы моделей для определения авторства, работающие на основе малого и большого количества данных
		\item Протестировать модель на доступных датасетах
	\end{itemize}
\end{frame}

\begin{frame}
	\frametitle{Сбор данных}
	\begin{itemize}
		\item Предлагаемый подход работает для произвольных проектов, использующих Git
		\item История проекта разбивается на коммиты, для них известен автор
		\item Коммиты разбиваются на изменения методов
		\item Изменения составляют датасет
		\item Подход реализован и собраны датасеты на основе IntelliJ IDEA
	\end{itemize}
	% Please add the following required packages to your document preamble:
% \usepackage{graphicx}
% \begin{table}[]
% 	\centering
% 	\resizebox{\textwidth}{!}{%
% 	\begin{tabular}{|c|c|c|c|c|}
% 	\hline
% 																 & \textbf{IDEA1} & \textbf{IDEA2} & \textbf{IDEA3} & \textbf{IDEA4} \\ \hline
% 	\textbf{Число авторов}         & 10             & 16             & 50             & 20             \\ \hline
% 	\textbf{Число примеров (тыс.)} & 912            & 648            & 1623           & 1228           \\ \hline
% 	\textbf{Макс. / Мин. примеров} & 5              & 3              & 32             & 8.5            \\ \hline
% 	\textbf{Примеров в группе}     & 1              & 16             & 16             & 16             \\ \hline
% 	\end{tabular}%
% 	}
% 	\end{table}
\end{frame}

\begin{frame}
	\frametitle{Сбор данных: датасеты}
	\begin{itemize}
		\item IDEA1-4: разный уровень сбалансированности и число авторов
		\item IDEA4-6: 20 авторов, деление по пакетам разного уровня
		\item IDEA7: 10 авторов разделение выборок по времени
	\end{itemize}
	\begin{table}[]
		\centering
		\resizebox{\textwidth}{!}{%
		\begin{tabular}{|c|c|c|c|c|}
		\hline
																	 & \textbf{IDEA1} & \textbf{IDEA2} & \textbf{IDEA3} & \textbf{IDEA4} \\ \hline
		\textbf{Число авторов}         & 10             & 16             & 50             & 20             \\ \hline
		\textbf{Число примеров (тыс.)} & 912            & 648            & 1623           & 1228           \\ \hline
		\textbf{Макс. / Мин. примеров} & 5              & 3              & 32             & 8.5            \\ \hline
		\textbf{Примеров в группе}     & 1              & 16             & 16             & 16             \\ \hline
		\end{tabular}%
		}
		\end{table}
\end{frame}

\begin{frame}
	\frametitle{Модель: мало данных}
	% \Large
	\begin{itemize}
		\item Случайный лес, число деревьев подобрано экспериментально и равно 500
		\item В качестве факторов используются частоты путей \footcite{alon2018general} и токенов в AST
		\item Факторы фильтруются на основе совместной информации между фактором и автором
		\item Для получения токенов и путей используется специально написанный инструмент, поддерживающий популярные языки программирования
		\item Произведено сравнение с имеющимися решениями
	\end{itemize}
\end{frame}

\begin{frame}
	\frametitle{Модель: много данных}
	\begin{itemize}
		\item Для работы адаптирована архитектура code2vec\footcite{alon2018code2vec}
		\item Модель уже улучшила результаты в определении имени метода и генерации описания по коду
		\item Архитектура расширена для работы с несколькими фрагментами кода
		\item Произведено тестирование на данных IntelliJ IDEA и существующих датасетах
	\end{itemize}
\begin{table}[]
	\centering
	\resizebox{\textwidth}{!}{%
	\begin{tabular}{|c|c|c|c|c|c|c|c|}
	\hline
										& \textbf{IDEA1} & \textbf{IDEA2} & \textbf{IDEA3} & \textbf{IDEA4}& \textbf{IDEA5} & \textbf{IDEA6} & \textbf{IDEA7} \\ \hline
	\textbf{Acc} & 74.4\%         & 99.7\%         & 93.5\%         & 97.8\% & 92\% & 87.3\% & 89.6\%         \\ \hline
	\textbf{MAP}      & 73.8\%         & 99.7\%         & 91.1\%         & 97.4\% & 93\% & 85.9\% & 88.5\%         \\ \hline
	\end{tabular}%
	}
	\end{table}
\end{frame}

\begin{frame}
	\frametitle{Модель: архитектура code2vec}
	% Представление кода через пути в AST\footcite{alon2018code2vec}, 94.2\% -- 16 метТестирование производов -- 50 разработчиков
	\begin{center}
		\includegraphics[width=1\textwidth]{images/VanillaCode2Vec-1.pdf}
	\end{center}
	\begin{center}
		\includegraphics[width=0.85\textwidth]{images/VanillaCode2Vec-2.pdf}
	\end{center}
\end{frame}

\begin{frame}
	\frametitle{Модель: модернизация code2vec}
	\begin{center}
		\includegraphics[width=1\textwidth]{images/AdoptedCode2Vec.pdf}
	\end{center}
\end{frame}

\begin{frame}
	\frametitle{Тестирование на существующих датасетах}
	\begin{itemize}
		\item Модель на основе случайного леса повторяет или улучшает результаты для трех языков
		\item Нейросетевая модель работает хуже из-за маленького размера обучающей выборки
	\end{itemize}
	\begin{table}[]
		\centering
		\resizebox{\textwidth}{!}                                                                    & 72.9\%                                                                               & -                                                                                  \\ \hline
		Alsulami, 2017 & -                                                                                   & 88.86\%                                                                              & -                                                                                  \\ \hline
		Yang, 2017    & -                                                                                   & -                                                                                    & 91.06\%                                                                            \\ \hline
		Эта работа, нейросеть      & 41.7\%                                                                    & -                                                                      & 86\%                                                                   \\ \hline
		Эта работа, случайный лес      & \textbf{92.7\%}                                                                    & \textbf{94.1\%}                                                                      & \textbf{97\%}                                                                   \\ \hline
		\end{tabular}%
		}
	\end{table}
\end{frame}

\begin{frame}
	\frametitle{Результаты}
	% \Large
	\begin{itemize}
		\item Реализован инструмент, генерирующий датасеты для определения авторства по проектам, использующим Git
		\item Создан набор из 7 датасетов изменений методов в IntelliJ IDEA
		\item Созданы и протестированы модели для работы с большим и ограниченным числом данных
		\item Модель на основе случайного леса повторяет или улучшает точность определения авторства для Java, Python и C++
		\item Часть работы представлена на MSR'19~\footcite{msr2019}
	\end{itemize}
\end{frame}

\begin{frame}[allowframebreaks]
\printbibliography[heading=none]
\end{frame}

\begin{frame}
	\frametitle{Результаты: IDEA}
	\begin{table}[]
		\centering
		\resizebox{\textwidth}{!}{%
		\begin{tabular}{|c|c|c|c|c|}
			\hline
									& \textbf{Точность} & \textbf{Acc\textsubscript{2}} & \textbf{Acc\textsubscript{5}} & \textbf{MAP} \\ \hline
			\textbf{IDEA1} & 74.4\%            & 86.3\%          & 95.7\%          & 74.3\%       \\ \hline
			\textbf{IDEA2} & 99.7\%            & 100\%           & 100\%           & 99.7\%       \\ \hline
			\textbf{IDEA3} & 94.2\%            & 97.2\%          & 99\%            & 91.9\%       \\ \hline
			\textbf{IDEA4} & 97.8\%            & 99.3\%          & 99.8\%          & 97.4\%       \\ \hline
			\textbf{IDEA5} & 92\%              & 96.5\%          & 99.1\%          & 93\%         \\ \hline
			\textbf{IDEA6} & 87.3\%            & 93.2\%          & 97.7\%          & 85.9\%       \\ \hline
			\textbf{IDEA7} & 89.6\%            & 95.6\%          & 98.9\%          & 88.5\%       \\ \hline
			\end{tabular}
		}
		\end{table}
\end{frame}

\begin{frame}
	\frametitle{Датасеты IDEA}
	\begin{table}[]
		\centering
		\resizebox{\textwidth}{!}{%
		\begin{tabular}{|c|c|c|c|c|}
		\hline
																	 & \textbf{IDEA1} & \textbf{IDEA2} & \textbf{IDEA3} & \textbf{IDEA4} \\ \hline
		\textbf{Число авторов}         & 10             & 16             & 50             & 20             \\ \hline
		\textbf{Число примеров (тыс.)} & 912            & 648            & 1623           & 1228           \\ \hline
		\textbf{Макс. / Мин. примеров} & 5              & 3              & 32             & 8.5            \\ \hline
		\textbf{Примеров в группе}     & 1              & 16             & 16             & 16             \\ \hline
		\end{tabular}%
		}
		\end{table}
	\begin{table}[ht]
		% \caption{Датасеты с разделением по времени и по пакетам разного уровня, построенные по проекту IntelliJ IDEA}
		\centering
		\begin{tabular}{|c|c|c|c|}
			\hline
										& \textbf{IDEA5} & \textbf{IDEA6} & \textbf{IDEA7} \\ \hline
			\textbf{Число авторов}         & 20   & 20   & 10  \\ \hline
			\textbf{Число примеров (тыс.)} & 1228 & 1228 & 912 \\ \hline
			\textbf{Макс. / Мин. примеров} & 8.5  & 8.5  & 5   \\ \hline
			\textbf{Примеров в группе}     & 16   & 16   & 16  \\ \hline
			\textbf{Деление по пакетам}    & 1    & 2    & 0   \\ \hline
			\textbf{Деление по времени}    & -    & -    & +   \\ \hline
			\end{tabular}%
		\label{tab:datasets-cool}
	\end{table}
	

\end{frame}

\begin{frame}
	\frametitle{Абстрактное синтаксическое дерево}
	\begin{center}
		\includegraphics[width=\textwidth]{images/ast-pbr.png}
	\end{center}
\end{frame}

\end{document}