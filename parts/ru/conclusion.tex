\section*{Заключение}

В рамках данной работы был описан TalkNet~\cite{beliaev2020talknet}, нейронная система синтеза речи основанная по конволюционных сетях. Модель состоит из двух сверточных сетей: предиктора длительностей графем и генератора мэл-спектрограмм. Эта модель не требует другой предобученной системы в качестве учителя. Истинное выравнивание графем извлекается из выходных данных CTC для предварительно обученной модели распознавания речи.

Предсказывание длительностей явным образом практически исключает возможность пропущенных или повторяющихся слов. TalkNet достигает сопоставимого уровня качества речи с Tacotron 2 и FastSpeech. Данный подход также обладает свойством компактности. Он имеет только $10.8$ миллионов обучаемых параметров, что почти в 3 раза меньше, чем предлагают аналогичные нейронные модели: Tacotron 2 имеет $28.2$ миллионов, а FastSpeech имеет $30.1$ миллионов параметров. Обучение TalkNet занимает всего около $2$ часов на сервере с 8 графическими ускорителями V100. Параллельная генерация мэл-спектрограммы делает скорость обучения и вывода значительно быстрее конкурентов.

Современные генеративные модели все еще обладают рядом фундаментальных проблем, которые не позволяют считать задачу генерации речи решенной. Ценность этой работы заключается не только в предложенном и описанном подходе, позволяющем сильно сократить время и количество параметров, требуемых TTS системе, но и в трудностях, возникших при реализации, тестировании и сравнении подходов, указывающих на глобальные проблемы и очерчивающих границы применимости того или иного метода или модели.

В рамках данной работы можно выделить следующие основные результаты:
\begin{itemize}
    \item Проанализирована предметная область и существующие модели. Обозначены основные проблемы и намечены пути к их решению.
    \item Разработана новая неавторегрессионая конволюционная архитектура, не требующая предобученной TTS системы в качестве учителя.
    \item Произведены сравнение подходов и анализ результатов на качество и скорость. Предложенный подход показал сравнимое качество, являясь при этом быстрее и компактнее аналогичных методов.
\end{itemize}

Представленный подход, несмотря на свою простоту, открывает сразу несколько направлений для дальнейшего изучения и доработки.
\begin{itemize}
    \item Основное направление - улучшение качества. Следуя аналогичным статьям, переход с графем на звуковые фонемы должен помочь модели правильнее произносить звуки в словах.
    \item TalkNet multi-speaker. Многоголосовое расширение позволит проще искать данные для обучения и увеличит число потенциальных применений в индустрии.
\end{itemize}

Модель, скрипт для обучения и сгенерированные примеры будут выложены с открытым исходным кодом как часть библиотеки NeMo~\cite{nemo}. Автор благодарят Джона Коэна, Виталия Лаврухина, Джейсона Ли, Кристофера Паризьена и Жоао Фелипе Сантоса за полезные отзывы и рецензии.