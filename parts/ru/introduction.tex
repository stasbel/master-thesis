\section*{Введение}

Нейронные сети и глубокое обучение позволили достигнуть результатов, сравнимых с теми, что может показывать человек, на различных задачах машинного обучения. Глубокие сети с residual соединениями~\cite{he2015deep} позволили обогнать человека на задаче классификации изображений. Трансформеры~\cite{attention-is-all} позволили научиться лучше понимать естественный язык и генерировать длинные отрывки текста подобно человеку. Генеративные состязательные сети~\cite{goodfellow2014generative} (GAN) позволили научиться генерировать изображения, не отличимые от оригинальных.

Однако, некоторые задачи остаются нерешенными до сих пор. Например, для задачи генерации аудио с речью по текстовому отрывку до сих пор не удается достигнуть качества голоса, не отличимого от записи настоящей человеческой речи. Более того, большинство подходов в области генерации речи требуют большое количество параметров и долго обучаются, что делает их труднодоступными для использования. Одна из дополнительных насущных проблем современных подходов -- это поддержание быстрой скорости генерации для возможности использования в realtime системах. Однако простые рекуррентные авторегрессионые подходы сильно ограничивают скорость с которой можно делать вывод.

Генерация речи имеет множество применений в современной жизни. Программы для синтеза используют повсеместно для голосовых помощников, перевода текстовой информации в аудио и систем для помощи инвалидам.

Таким образом, целью данной работы является разработка модели машинного обучения, позволяющей производить эффективную, качественную и быструю генерацию речи из входного текста. В рамках данной работы, решение будет основываться на сверточных (конволюционных) сетях с неавторегрессионной архитектурой, позволяющей рассчитывать на максимальную производительность на современных графических ускорителях (GPU).

Для достижения описанной выше цели необходимо решить следующие задачи:
\begin{itemize}
    \item Проанализировать предметную область и существующие модели. Обозначить основные проблемы и пути к их решению.
    \item Разработать и описать эффективную архитектуру, основанную на идеи неавторегрессионности.
    \item Выбрать данные для обучения и провести эксперименты.
    \item Произвести сравнение подходов и анализ результатов на качество и скорость.
\end{itemize}

В рамках данной работы предлагается TalkNet: сверточная неавторегрессионная нейронная модель, которая решает задачу синтеза речи. Модель состоит из двух прямых (feed-forward) полностью сверточных нейронных сетей. Первая сеть служит для предсказания длительности входных символов (графем), выравнивая таким образом входную последовательность на длину мэл-спектрограммы. Далее, производится операция расширения (expansion) входного текста путем повторения каждого символа в соответствии с предсказанной длительностью. Вторая сеть генерирует мэл-спектрограмму из развернутого текста. Операция разворачивания, таким образом, позволяет построить неавторегрессионную архитектуру.

Чтобы обучить предиктор длительностей графем, истинные длительности для набора данных для обучения получаются из предварительно обученной модели для распознавания речи на основе Connectionist Temporal Classification (CTC) функции ошибки. Явное предсказание длительностей исключает пропуски и повторения слов. Эксперименты с набором данных LJSpeech показывают, что качество речи TalkNet сравнимо с авторегрессионными подходами. Модель очень компактна -- она имеет всего около $10.8$ миллионов обучаемых параметров, что почти в 3 раза меньше, чем предлагают современные модели преобразования текста в речь. Неавторегрессионная архитектура также позволяет быстро обучаться и делать выводы.

В главе $1$ будут описана формальная постановка задачи генерации речи и существующие подходы к ее решению, а также их недостатки и достоинства. В главе $2$ происходит описание главной идеи TalkNet с разбиением процесса генерации на два шага. Глава $3$ посвящена описанию данных и проводимых экспериментов, а также подробностям процесса обучения. В главе $4$ проводится анализ результатов с точки зрения качества и скорости.