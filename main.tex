\documentclass{spbau-diploma}

\newcommand\TODO[1]{\textcolor{red}{TODO: #1}}  % To mark TODOs.
\overfullrule=3mm  % To note hbox errors.
\newcommand\sep{\newline\textcolor{red}{\rule{\textwidth}{3pt}}\newline\indent}  % RU-EN separator

% Chars count
\usepackage{shellesc}
\usepackage{lastpage}
\newread\tmp
\newcommand{\stats}[1]{
\ShellEscape{texcount -1 -sum=1,1,1,1,1,1,1 -merge -char -q #1.tex output.bbl > #1-chars.sum}
\openin\tmp=#1-chars.sum%
\read\tmp to \chars%
\closein\tmp%
\ShellEscape{texcount -1 -sum=1,1,1,1,1,1,1 -merge -q #1.tex output.bbl > #1-words.sum}
\openin\tmp=#1-words.sum%
\read\tmp to \words%
\closein\tmp%
\noindent\textcolor{red}{\Huge Chars: \the\numexpr \words + \chars \relax \slash 40000}\\
\textcolor{red}{\Huge Pages: \pageref*{LastPage}\slash 40}
}

% В рамках данной работы предлагается TalkNet: сверточная неавторегрессионная нейронная модель, которая решает задачу синтеза речи. Модель состоит из двух прямых (feed-forward) полностью сверточных нейронных сетей. Первая сеть служит для предсказания длительности входных символов (графем), выравнивая таким образом входную последовательность на длину мэл-спектрограммы. Далее, производится операция расширения (expansion) входного текста путем повторения каждого символа в соответствии с предсказанной длительностью. Вторая сеть генерирует мэл-спектрограмму из развернутого текста. Операция разворачивания, таким образом, позволяет построить неавторегрессионную архитектуру. Эксперименты с набором данных LJSpeech показывают, что качество речи TalkNet сравнимо с авторегрессионными подходами. Модель очень компактна -- она имеет всего около 10.8 миллионов обучаемых параметров, что почти в 3 раза меньше, чем предлагают современные модели преобразования текста в речь. Неавторегрессионная архитектура также позволяет быстро обучаться и делать выводы.
% This work propose TalkNet, a convolutional non-autoregressive neural model for speech synthesis. The model consists of two feed-forward convolutional networks. The first network predicts grapheme durations. An input text is expanded by repeating each symbol according to the predicted duration. The second network generates a mel-spectrogram from the expanded text. To train a grapheme duration predictor, the grapheme duration of the training dataset extracted using a pre-trained Connectionist Temporal Classification (CTC)-based speech recognition model. The explicit duration prediction eliminates word skipping and repeating. Experiments on the LJSpeech dataset show that the speech quality nearly matches auto-regressive models. The model is very compact -- it has 10.8M parameters, almost 3x less than the present state-of-the-art text-to-speech models. The non-autoregressive architecture allows for fast training and inference.

\begin{document}

% \stats{main}

\filltitle{ru}{
    chair              = {Кафедра математических и информационных технологий},
    title              = {Пустое подмножество как замкнутое множество},
    type               = {master},
    position           = {студента},
    group              = 666,
    author             = {Беляев Станислав Валерьевич},
    supervisorPosition = {д.\,ф.-м.\,н., профессор},
    supervisor         = {Выбегалло А.\,А.},
    reviewerPosition   = {ст. преп.},
    reviewer           = {Привалов А.\,И.},
    chairHeadPosition  = {д.\,ф.-м.\,н., профессор},
    chairHead          = {Омельченко А.\,В.},
    % university = {САНКТ-ПЕТЕРБУРГСКИЙ АКАДЕМИЧЕСКИЙ УНИВЕРСИТЕТ},
    % faculty = {Центр высшего образования},
    % city = {Санкт-Петербург},
    % year             = {2013}
}
\filltitle{en}{
    chair              = {Department of Mathematics and Information Technology},
    title              = {Empty subset as closed set},
    author             = {Edelweis Mashkin},
    supervisorPosition = {professor},
    supervisor         = {Amvrosy Vibegallo},
    reviewerPosition   = {assistant},
    reviewer           = {Alexander Privalov},
    chairHeadPosition  = {professor},
    chairHead          = {Alexander Omelchenko},
}
\maketitle
\tableofcontents

\section*{Введение}

Современная , однако одтельные области машинного обучения развиваются медленно или недостаточно быстро.

\begin{figure}[!ht]
\centering
\includegraphics[width=1.0\textwidth]{images/arch.png}
\caption{TalkNet converts text to speech, using a grapheme duration predictor, a mel-spectrogram generator, and a vocoder. We use $\sim$ to denote the CTC blank symbol.}
\label{fig:arch}
\end{figure}

Таким образом, целью данной работы является разработка модели машинного обучения, позволяющей производить эффективную, качественную и быструю генерацию речи из входного текста. В рамках данной работы, решение будет основываться на сверточных (конволюционных) сетях с неавторегрессионной архитектурой, позволяющей расчитывать на максимальную производительность на современных графических ускорителях (GPU).

Для достижения описанной выше цели необходимо решить следующие задачи:
\begin{itemize}
    \item Проанализировать предметную область и существующие модели. Обозначить основные проблемы и пути к их решению.
    \item Разработать и описать эффективную архитектуру, основанную на идеи неавторегрессионности.
    \item Выбрать данные для обучения и провести эксперименты.
    \item Произвести сравнение подходов и анализ результатов на качество и скорость.
\end{itemize}

В рамках данной работы мы предлагаем TalkNet: сверточную неавторегрессионную нейронную модель для задачи синтеза речи. Модель состоит из двух прямых (feed-forward) полностью сверточных сетей. Первая сеть предсказывает длительность входных символов (графем), выравнивая таким образом входную последовательность на длинну мэл-спеткрограммы. Далее, производится операция расширения (expansion) входного текста путем повторения каждого символа в соответствии с предсказанной длительностью. Вторая сеть генерирует мэл-спектрограмму из развернутого текста. Операция разворачивания, таким образом, позволяет построить неавторегрессионную архитектуру.

Чтобы обучить предиктору длительности графемы, мы добавляем длительность графемы в обучающий набор данных, используя предварительно обученную модель распознавания речи на основе Коннекционистской временной классификации (CTC). Явное предсказание длительности исключает пропуск и повторение слов. Эксперименты с набором данных речи ЖЖ показывают, что качество речи почти соответствует авторегрессионным моделям. Модель очень компактна - она имеет параметры 10,8 м, что почти в 3 раза меньше, чем современные модели преобразования текста в речь. Неавторегрессионная архитектура позволяет быстро обучаться и делать выводы.
\sep
To train a grapheme duration predictor, we add the grapheme duration to the training dataset using a pre-trained Connectionist Temporal Classification (CTC)-based speech recognition model. The explicit duration prediction eliminates word skipping and repeating. Experiments on the LJSpeech dataset show that the speech quality nearly matches auto-regressive models. The model is very compact -- it has 10.8M parameters, almost 3x less than the present state-of-the-art text-to-speech models. The non-autoregressive architecture allows for fast training and inference.

В главе $1$ будут описана формальная постановка задачи генерации речи и существующие подходы к ее решению, а также их недостатки и достоинства. В главе $2$ происходит описание главной идеи TalkNet с разбиением процесса генерации на два шага. Глава $3$ посвящена описанию данных и проводимых экспериментов, а также подробностях процесса обучения. В главе $4$ проводится анализ результатов с точки зрения качества и скорости.  % *
\section{Обзор предметной области}

\subsection{Постановка задачи}

Задача генерации речи имеет простую математическую формулировку. По заданному отрывку текста со строковым представлением (конечная последовательность символов из конечного алфавита), сгенерировать аудиодорожку с хорошо слышимой человеческой речью соответствующей заданному отрывку. Формат аудио обычно представляет из себя последовательность из 16-битных действительных чисел от $-1.0$ до $1.0$ -- дискретное приближение непрерывной аудиоволны. Основной параметр такого формата - это частота дискретизации, выражаемая в герцах (Гц, Hz). Популярными частотами для аудио отрывков в системах генерации речи являются $22.05$ кГц и $24$ кГц (22050 и 24000 значения в секунду соответственно). Чем выше частота дискретизации -- тем лучше качество аудио (так как лучше приближение), но тем сложнее уловить зависимости для генерации (Рисунок~\ref{fig:sample-rate}).

\begin{figure}[!ht]
\centering
\includegraphics[width=1.0\textwidth]{images/sample-rate.png}
\caption{Примеры различной частота дискретизации (sample rate) для представления аудио}
\label{fig:sample-rate}
\end{figure}

Модели на основе нейронных сетей (NN) для преобразования текста в речь (Text-To-Speech, TTS) превзошли как конкатенативный (concatenative), так и статистический параметрический подходы для синтеза речи с точки зрения качества. Они также значительно упрощают процесс синтеза речи. Традиционно. системы синтеза речи объединяет несколько блоков: модель для извлечения лингвистических признаков из текста, модель предсказания длительности, модель предсказания акустических признаков и вокодер на основе обработки сигналов~\cite{taylor}, который служит для преобразования акустических признаков в аудиодорожку. Нейронные TTS системы, как правило, имеют два этапа (Рисунок~\ref{fig:tts-pipeline}). На первом этапе модель генерирует мэл-спектрограммы из текста. На втором этапе вокодер на основе нейронной сети синтезирует речь из мэл-спектрограмм. Большинство моделей TTS на основе нейронных сетей имеют архитектуру encoder-decoder~\cite{bahdanau} с операциями с механизмом внимания (attention), которые, как было замечено, имеют некоторые общие проблемы:
\begin{enumerate}
    \item Тенденция повторять или пропускать слова~\cite{fastspeech}, из-за сбоев работы механизма внимания, когда некоторые подпоследовательности повторяются или игнорируются. Для решения этой проблемы модели, основанные на механизме внимания, используют дополнительные правила поощрения монотонного внимания~\cite{tacotron2, deepvoice3, taigman2017}. В общем же случае, механизм внимания часто ведет к проблемам на этапа вывода, а также замедляет скорость обучения, так как обычно состоит как минимум из одной операции с квадратичной по времени асимптотикой.
    \item Медленная скорость вывода относительно параметрических моделей.
    \item Нет простого способа контролировать просодию (паттерн ритма и интонации голоса) или скорость голоса, так как длина генерируемой последовательности определяется декодером.
\end{enumerate}

\begin{figure}[!ht]
\centering
\includegraphics[width=1.0\textwidth]{images/tts-pipeline.png}
\caption{Два шага TTS систем. Первый шаг -- генерация мэл-спектрограммы -- будет рассматриваться в рамках данной работы. Второй шаг -- вокодинг -- отдельная сложная задача. Оба шага имеют полно нерешенных проблем: медленная скорость работы, большое количество весов, заметно уступающее качество в сравнении с записью человеческой речи.}
\label{fig:tts-pipeline}
\end{figure}

\subsection{Особенности предлагаемого подхода}

В рамках данной работы описывается новая нейронная модель TTS для решения проблем описанных выше. Модель состоит из двух сверточных сетей. Первая сеть предсказывает длительности графем (входных символов). Далее, входной текст расширяется, повторяя каждый символ в соответствии с предсказанной длительностью. Вторая сеть генерирует мэл-спектрограммы из развернутого текста. Наконец, используется вокодер WaveGlow~\cite{waveglow} для синтеза звука из мэл-спектрограмм (Рисунок~\ref{fig:arch}). Финальная часть -- вокодер -- не является частью модели и должна рассматриваться в рамках отдельной задачи, но в данной работе будет использоваться WaveGlow из-за наличия универсальной претренированой модели и возможности сравнения с другими подходами.

\begin{figure}[!ht]
\centering
\includegraphics[width=1.0\textwidth]{images/arch.png}
\caption{TalkNet преобразует текст в речь, используя предиктор длительностей графем, генератор мэл-спектрограмм и вокодер. $\sim$ используется в рамках данной работы для обозначения пустого символа из выхода CTC функции ошибки.}
\label{fig:arch}
\end{figure}

Чтобы обучить предсказатель длительностей графем, нам нужно сначала получить истинное выравнивание между входными символами и звуковой дорожкой по времени. Аналогичная проблема выравнивания существует в автоматическом распознавании речи (ASR), которая решается явным образом с помощью Connectionist Temporal Classification (CTC)~\ref{fig:ctc}. CTC маргинализирует вывод по всем возможным выравниваниям, выбирая наилучшее. Выбирая наиболее вероятный выход в каждый момент времени, его можно использовать его для выравнивания между входным звуком и выходным текстом. Это выравнивание не является совершенным, и в нем могут быть ошибки. В рамках данной работы показано, что если модель ASR точна и имеет низкую частоту ошибок в символах (Char-Error-Rate, CER), то можно извлечь достаточно хорошее выравнивание между текстом и звуком. Полученное выравнивание на основе CTC можно применить для обучения модели, которая будет предсказывать длительности графем для входного текста. Предиктор длительности графемы заменяет выравнивание на основе механизма внимания (attention) и позволяет избежать пропуски и повторения слов. При этом. эксперименты с набором данных LJSpeech~\cite{ljspeech} показывают, что качество речи для TalkNet сравнимо с лучшими авторегрессионным подходами.

Конволюционная структура обоих частей позволяет проводить параллельное обучение, также значительно ускоряет скорость вывода. Такая структура позволяет работать значительно быстрее со значительно меньшим числом параметров, поддерживая качество генерируемой речи аналогичное FastSpeech~\cite{fastspeech} и Tacotron 2~\cite{tacotron2}.

\subsection{Обзор методов генерации речи}

Методы синтеза речи, основанные на статистике, обычно имеют следующие части: преобразователь графем в фонемы, предиктор длительностей фонемы, генератор акустических признаков (например, мэл-спектрограмм) и вокодер~\cite{taylor}. Zen et al~\cite{zen2009,zen-2015,zen-2016} предложили гибридную нейронную параметрическую модель TTS (Рисунок~\ref{fig:stats-tts} и \ref{fig:zen-tts}), в которой глубокие нейронные сети используются для прогнозирования длительности фонемы и генерации акустических признаков на уровне кадра. Предиктор длительности фонемы был обучен на основе скрытой марковской модели (HMM), из которой извлекались фонетические выравнивания.

\begin{figure}[!ht]
\centering
\includegraphics[width=1.0\textwidth]{images/related-work/stats-tts.png}
\caption{Блок-диаграмма работы статистической TTS системы с предсказанием длительностей графем~\cite{zen2009}}
\label{fig:stats-tts}
\end{figure}

\begin{figure}[!ht]
\centering
\includegraphics[width=1.0\textwidth]{images/related-work/zen-tts.png}
\caption{Более современный вариант модели с предсказыванием длительностей входных графем с использованием нейронных сеток~\cite{zen-2015,zen-2016}}
\label{fig:zen-tts}
\end{figure}

Один из других возможных подходов -- DeepVoice~\cite{deepvoice1, deepvoice2} (Рисунок~\ref{fig:related-work:deepvoice-1} и \ref{fig:related-work:deepvoice-2}) -- также основан на для традиционной для области синтеза речи структуре, но заменяет обучаемые компоненты на нейронные сети. Для обучения предиктора длительности фонем была использована вспомогательная модель на основе CTC для фонетической сегментации, позволяющая аннотировать данные по границам фонем. Другая модель -- Tacotron~\cite{tacotron1,tacotron2} (Рисунок~\ref{fig:related-work:tacotron-2}) -- это end-to-end нейронная сеть, которая принимает символы в качестве входных данных и сразу выводит мэл-спектрограмму слева направо шаг за шагом (авторегрессионно). Tacotron 2 использует архитектуру encoder-decoder с механизмами внимания. Encoder состоит из трех сверточных слоев и одного двунаправленного LSTM. Decoder представляет собой рекуррентную нейронную сеть (RNN) с чувствительным к местоположению монотонным вниманием (location-sensitive monotonic attention). Авторегрессионность Tacotron2 позволяет хорошо учитывать контекст, а механизмы внимания позволяют правильно расставлять акценты при произношении. Tacotron2 -- одна из лучших на сегодняшний день моделей по качеству, однако для ее обучения требуется много времени (до нескольких дней).

\begin{figure}[!ht]
\centering
\includegraphics[width=1.0\textwidth]{images/related-work/deepvoice-1.png}
\caption{Диаграмма описывающая модель DeepVoice1~\cite{deepvoice1} с промежуточным шагов в виде предсказания длительностей}
\label{fig:related-work:deepvoice-1}
\end{figure}

\begin{figure}[!ht]
\centering
\includegraphics[width=1.0\textwidth]{images/related-work/deepvoice-2.png}
\caption{DeepVoice2 с multi-speaker расширением}
\label{fig:related-work:deepvoice-2}
\end{figure}

\begin{figure}[!ht]
\centering
\includegraphics[width=0.7\textwidth]{images/related-work/tacotron-2.png}
\caption{Процесс генерации речи для авторегрессионного Tacotron2}
\label{fig:related-work:tacotron-2}
\end{figure}

Последовательный характер моделей на основе рекуррентных нейронных сетей (RNN) ограничивает эффективность обучения и вывода. Но для генерации речи не обязательно использовать RNN. DeepVoice 3~\cite{deepvoice3} (Рисунок~\ref{fig:related-work:deepvoice-3}) заменяет RNN на конволюционную модель с encoder-decoder архитектурой и монотонным механизмом внимания. Переход от RNN к сверточной нейронной сети (CNN) ускоряет обучение, но вывод модели по-прежнему является авторегрессионным. Другой end-to-end моделью TTS, которая не использует RNN, является ParaNet~\cite{paranet} (Рисунок~\ref{fig:related-work:paranet}). ParaNet -- это еще одна конволюционная encoder-decoder с механизмом внимания. Для обучения ParaNet требуется другая предобученная TTS модель, у которой заимствуется матрица внимания. Наконец, Transformer-TTS и~\cite{transformer-tts} (Рисунок~\ref{fig:related-work:transformer-tts}) заменяет рекуррентные нейронные сети с encoder-decoder структурой на трансформеры~\cite{attention-is-all} (модель, основанная на применении механизма внимания несколько раз подряд). Transformer-TTS сначала преобразует текст в фонемы с помощью конвертера на основе правил. Используя последовательности фонем в качестве входных данных, Transformer-TTS генерирует мэл-спектрограмму.

\begin{figure}[!ht]
\centering
\includegraphics[width=1.0\textwidth]{images/related-work/deepvoice-3.png}
\caption{Encoder-decoder структура в архитектуре DeepVoice3}
\label{fig:related-work:deepvoice-3}
\end{figure}

\begin{figure}[!ht]
\centering
\includegraphics[width=1.0\textwidth]{images/related-work/paranet.png}
\caption{Архитектура ParaNet~\cite{paranet}}
\label{fig:related-work:paranet}
\end{figure}

\begin{figure}[!ht]
\centering
\includegraphics[width=0.7\textwidth]{images/related-work/transformer-tts.png}
\caption{Encoder-decoder Архитектура Transformer-TTS~\cite{transformer-tts}}
\label{fig:related-work:transformer-tts}
\end{figure}

Как и в других моделях, основанных на внимании, Tacotron, Transformer-TTS и ParaNet иногда пропускают или повторяют слова~\cite{paranet}. Чтобы предотвратить пропуск и повторение слов, FastSpeech~\cite{fastspeech} предлагает end-to-end модель с использованием трансформеров~\cite{attention-is-all} взамен традиционной структуре encoder-attention-decoder. FastSpeech использует явный регулятор длины в виде отдельного декодера, который расширяет последовательность фонем в соответствии с предсказанной длительностью, чтобы длина последовательности стала соответствовать длине мэл-спектро\-граммы. Длительности фонем извлекается из выравнивания механизма внимания во внешней предварительно обученной модели TTS, Tacotron 2.
Такой подход показывает неплохие результаты с точки зрения качества и значительно повышает скорость генерации. FastSpeech -- первая модель, показавшая, что идею с промежуточным шагов предсказания длительностей можно успешно применять для построения хорошо работающей TTS системы. Во многом, результаты текущей работы, основаны на продолжении и развитии идей FastSpeech.

\begin{figure}[!ht]
\centering
\includegraphics[width=1.0\textwidth]{images/related-work/fastspeech.png}
\caption{Архитектура FastSpeech~\cite{fastspeech}}
\label{fig:related-work:fastspeech}
\end{figure}

Как видно, идеи из области глубокого обучения позволили успростить подходы для генерации речи.  % 1
\section{Описание подхода}

TalkNet разбивает генерацию мэл-спектрограммы из текста на два отдельных модуля. Первый модуль, предсказатель длительности, выравнивает входные графемы по времени относительно звуковой дорожки (или мэл-спектрограммы, что то же самое так так длина мэл-спектрограммы линейно зависит от длины аудио). Второй модуль, генератор мэл-спектрограмм, производит генерацию из выровненных по времени входных символов (Рисунок~\ref{fig:arch}). Были использованы конволюционные модели прямого вывода (feed-forward) для обоих модулей, поэтому как обучение, так и вывод не являются авторегрессионными. Это позволяет гораздо быстрее обучаться и делать вывод по сравнению с авторегрессионными подходами. Для обучения предсказателя, истинные длительности графем извлекались из выхода CTC для предварительно обученной модели распознавания речи (Automatic-Speech-Recognition, ASR).

Таким образом, вводя дополнительный шаг на пути к получению мэл-спектро\-грамм, появляется возможность явным образом контролировать манеру произношения (просодию). Длительности букв можно изменить вручную, указав где нужно сделать паузу, а где наоборот проговорить быстро. Обе части могут обучаться независимо и параллельно, поэтому такой переход от end-to-end архитектуре к нескольким шагам оправдан с точки зрения времени.

\subsection{Извлечение истинных длительностей графем}

Центральная идея TalkNet заключается в использовании модели ASR на основе Connectionist Temporal Classification (CTC) функции ошибки для извлечения выравниваний графем. CTC присваивает вероятность каждому из символов алфавита и использует вспомогательный пустой символ $\sim$. Первым шагом схлопываются соседние повторявшиеся символы в выводе, подсчитывая таким образом длительность каждого символа. Пустой символ выступает как промежуточное состояние между двумя соседними графемами, и его длительность соответствует длительности перехода от одного символа к другому. Для каждого временного шага выбирается наиболее вероятный символ из выходных данных CTC (Рисунок~\ref{fig:ctc}).

\begin{figure}[!ht]
\centering
\includegraphics[width=1.0\textwidth]{images/snippets/ctc.png}
\caption{Пример работы CTC алгоритма~\cite{hannun2017sequence}. Соседние буквы схлопываются, а $\epsilon$ ($\sim$ в нашей нотации) служит разделительным вспомогательным символом.}
\label{fig:ctc}
\end{figure}

CTC -- это функция ошибки, используемая для этапа обучения. Поэтому, выход CTC часто бывает неточен. Однако, задача ASR решена намного лучше TTS, поэтому ошибка все еще намного меньше. Выход CTC выравнивается с истинным отрывком текста для того чтобы максимально устранить влияние ошибки. Была использована функция \textit{pairwise2} из пакета Biopython~\cite{biopython} (пакет для использования в биоинформатике для языка Python), которая выравнивает два строковых представления побуквенно, используя наименьшее количество операции добавления и удаления символов. Затем удаляются все неправильные символы в выводе CTC, а их длительность добавляется к ближайшему пустому, а также добавляются недостающие символы с длительность $0$. Затем, все символы с предсказанной длительностью 0 получают длительность 1, вычитая 1 из почти самого большого соседнего $\sim$, чтобы сумма всех длительностей графемы была равна длине мэл-спектрограммы (Рисунок~\ref{fig:alignment}).

\begin{figure}[!ht]
\centering
\includegraphics[width=1.0\textwidth]{images/alignment.png}
\caption{Извлечение длительности графемы из вывода CTC. $\sim$ используется для обозначения пустого символа в CTC.}
\label{fig:alignment}
\end{figure}

В качестве модели с CTC выводом для задачи распознавания текста (ASR) используется QuartzNet~\cite{quartznet}. QuartzNet (Рисунок~\ref{fig:qn}) -- это полностью коволюционная нейронная архитектура, основными достоинствами которой являются:
\begin{itemize}
    \item Низкое количество параметров (около 18 миллионов), которое было достигнуто за счет использования depthwise separable~\cite{kaiser2017depthwise} конволюций, являющихся математическим приближение обычных конволюций (Рисунок~\ref{fig:dws-conv}).
    \item Простая неавторегрессионная архитектура с базовыми операциями из глубокого обучения (конволюции, нелинейности, батч-нормализация и дропаут), позволяющяя значительно ускорить процесс обучения и вывода (inference).
    \item CTC функция в качестве функции ошибки в декодере, которая не содержит дополнительных параметров и обеспечивает быстрый вывод.
\end{itemize}

\begin{figure}[!ht]
\centering
\includegraphics[width=1.0\textwidth]{images/snippets/dws-conv-1.png}
\includegraphics[width=1.0\textwidth]{images/snippets/dws-conv-2.png}
\caption{Depthwise separable конволюции. Применяется в два этапа: на первом используется 1d конволюции по времени, на втором применяются $1x1$ pointwise свертки. Два шага работают сообща и действуют как аппроксимация обычных сверток к квадратичными ядрами. В TalkNet такие свертки реализованы напрямую, через две последовательные конволюции.}
\label{fig:dws-conv}
\end{figure}

\begin{figure}[!ht]
\centering
\includegraphics[width=0.8\textwidth]{images/qn.png}
\caption{Оригинальная архитектура QuartzNet 15x5}
\label{fig:qn}
\end{figure}

Для получения истинных длительностей графем была использована архитектура QuartzNet 15x5 (15 блоков по 5 повторений). Выход такой модели по длине в 2 раза меньше входной мэл-спектрограммы. Причина -- в самой первой конволюции выставлен параметр $\texttt{stride}=2$ (Рисунок~\ref{fig:stride-2}). QuartzNet использует удвоение шага в самом начале, так как для любого примера длина выходного текста как минимум в два раза меньше длины мэл-спектрограммы, поэтому такой трюк позволяет сократить количество вычислений вдвое. Однако, это так же уменьшает длительность каждого символа в выходе CTC. Чтобы сравнять сумму длительностей с длиной мэла, QuartzNet претерпела изменения, устанавливая $\texttt{stride}=1$ для первого слоя. Заметим так же что это не убирает возможность воспользоваться предобученной моделью, загрузив веса перед обучением для дообучения (fine-tuning) -- размеры, форма и количество ядер (kernels) конволюций остается неизменным. Соответственно, не изменяются и размеры матриц и векторов с весами.

\begin{figure}[!ht]
\centering
\includegraphics[width=0.8\textwidth]{images/snippets/stride-2.png}
\caption{Пример работы конволюций с удвоенным шагом}
\label{fig:stride-2}
\end{figure}

QuartzNet 15x5 дообучался на данных из датасета LibriTTS~\cite{libritts}. LibriTTS это набор данных из того же источника, что и LibriSpeech, на котором успешно обучался оригинальный QuartzNet. Однако, LibriTTS использует другую обработку данных, которая более подходит для задач генерации речи, нежели распознавания речи. В частности, LibriTTS обрезает отрывки аудио по большим паузам, оставляет всю пунктуацию нетронутой (для экспрессивности речи), а также разворачиваем некоторые числа и буквенные сокращения. Для токенизации входного текста (разбиения на символы) была оставлена вся пунктуацию, давая возможность CTC самому назначит длительность каждому символу. При дообучении QuartzNet на LibriTTS достигалась побуквенная ошибка (Char-Error-Rate, CER) порядка $4.51\%$ на части dev-clean и порядка $3.54\%$ на тестовой части LJSpeech~\cite{ljspeech}. Выравнивание, полученное из CTC, используется для обучения предиктора длительности графемы.

Таким образом, вместо того чтобы использовать другую преодобученную TTS модель в качестве учителя для получения длительностей графем, как это делалось в модели FastSpeech~\cite{fastspeech}, в рамках данной работы представлен метод в котором используется ASR модель. Ошибка, получаемая в CTC гораздо меньше ошибки при генерации речи, поэтому такой способ позволяет снять жесткое ограничение на качество, задаваемой моделью учителя.

\subsection{Предсказатель длительностей графем}

Первая часть TalkNet'а служит для предсказания длины мэл-спектрограммы с помощью соответствия каждому входному символу (включая пунктуацию) количество единиц времени, требуемых для их вывода. Первым шагом предиктор длительностей вставляет пустой символ $\sim$ между каждыми двумя соседними символами. Затем он предсказывает длительность для каждого входного символа с помощью конволюционной нейронной архитектуры. Далее, производится операция расширения (expansion) входных символов в соответствии с предсказанной длительностью (Рисунок~\ref{fig:durs}).

\begin{figure}[!ht]
\centering
\includegraphics[width=1.0\linewidth]{images/durs.png}
\caption{Процесс предсказания длительностей графем}
\label{fig:durs}
\end{figure}

Модель предиктора длительностей представляет собой конволюционную нейронную сеть, основанную на архитектуре модели для распознавания речи QuartzNet~\cite{quartznet}. Модель имеет $5$ больших блоков с 5 повторениями на блок. Подблок состоит из depthwise separable~\ref{fig:dws-conv} конволюции, батч нормализации, нелинейности ReLU и дропаута (Рисунок~\ref{fig:qn-block}). Помимо этого, применяются два дополнительных слоя: обучаемая векторизация токенов графем и $1x1$ слой перед передачей в функцию потери (Таблица~\ref{tab:durs-model}). Размерность последнего слоя зависит от типа функции потерь: для $L_2$ это 1, для кросс-энтропии это $|\texttt{множество\_классов}|$.

\begin{figure}[!ht]
\centering
\includegraphics[width=0.6\linewidth]{images/qn-block.png}
\caption{Базовый блок QuartzNet. Как предиктор длительностей графем, так и генератор мэл-спектрограмм являются сверточными сетями с 1D time-channel свертками на основе QuartzNet~\cite{quartznet}.}
\label{fig:qn-block}
\end{figure}

Тренировка предиктора длительностей происходит при использовании $L_2$ функции ошибки с логарифмированием целевых значений аналогично~\cite{fastspeech}. Таким образом, больший вес при обучении присваивается символам с меньшими длительностями. В самом деле, разница между $15$ и $16$ не такая значительная как между $1$ и $2$. Была также испробована другая функция ошибки -- кросс-энтропийные критерий, где каждый класс соответствовал определенной длительности. При классификации, были выбраны $32$ самых частых класса (первые $32$ длительности -- от $0$ до $31$), а также добавлены редкие большие длительности с логарифмическим шагом после $32$, так как распределение длительности графемы имеет длинный хвост (Рисунок~\ref{fig:durs-dist}). Как можно видеть, кросс-энтропия имеет несколько более высокую поклассовую точность (Таблица~\ref{tab:dur-results}). Однако, в рамках данной работы будет использоваться $L2$, так как речь, сгенерированная с меньшим MSE для длительностей, получила несколько более высокий mean opinion score (MOS) в процессе валидации.

\begin{figure}[!ht]
\centering
\includegraphics[width=.48\linewidth]{images/durs-dist.png}
\includegraphics[width=.48\linewidth]{images/blanks-dist.png}
\caption{Распределение длительностей исходных символов (слева) и символов перехода (справа) на основе вывода CTC для набора данных LJSpeech. Максимальная длительность для символов составляет $7$, а для $\sim$ - $493$.}
\label{fig:durs-dist}
\end{figure}

\begin{table}[!ht]
\centering
\scalebox{1.2}{
\begin{tabular}{c c c c c} 
\toprule
\textbf{Block} &
\textbf{\thead{\# Sub\\Blocks}} &
\textbf{\thead{\# Output\\Channels}} &
\textbf{Kernel Size} &
\textbf{Dropout} \\
\midrule
Embed & 1 & 64  & 1 & 0.0  \\
Conv1 & 3 & 256 & 3 & 0.1  \\
$B_1$ & 5 & 256 & 5 & 0.1  \\
$B_2$ & 5 & 256 & 7 & 0.1  \\
$B_3$ & 5 & 256 & 9 & 0.1  \\
$B_4$ & 5 & 256 & 11 & 0.1 \\
$B_5$ & 5 & 256 & 13 & 0.1 \\
Conv2 & 1 & 512 & 1 & 0.1  \\
Conv3 & 1 & $32$ & 1 & 0.0 \\
\midrule
\textbf{Params, M} & & & & \textbf{2.3} \\
\bottomrule
\end{tabular}
}
\caption{Предиктор длительностей графем основан архитектуре QuartzNet 5х5. Residual соединения и увеличивающиеся размеры сверток позволяют эффективно выучивать различнные паттерны и комбинировать их на поздних слоях.}
\label{tab:durs-model}
\end{table}

\begin{table}[!ht]
\centering
\scalebox{1.2}{
\begin{tabular}{c c c c c c c} 
\toprule
\textbf{Method} &
\textbf{MSE} &
\textbf{Accuracy, $\%$} &
$\mathbf{|P - T| \leq 1}$ &
$\mathbf{|P - T| \leq 3}$\\
\midrule
$L_2$ & 7.81 & 67.69 & 91.90 & 97.17 \\
XE & 10.46 & 69.42 & 92.90 & 97.40 \\
\bottomrule
\end{tabular}
}
\caption{Результаты предиктора длительностей на тестовой части LJSpeech. $P$ - предсказание, $T$ - целевое значение.}
\label{tab:durs-results}
\end{table}

\subsection{Генератор мэл-спектрограмм}

Второй модуль производит генерацию мэл-спектрограммы из развернутого текста. Генератор представляет собой сверточную сеть, также основанную на архитектуре QuartzNet. Он имеет 9 блоков с 5 подблоками (Таблица\ref{tab:mels-model}). Как можно видеть, размер ядер конволюций увеличивается от слоя к слою с 5 до 25. Это, вкупе с residual ребрами вычислительного графа, позволяют модели выучить паттерны разного размера для применения на входной последовательности и объединить их на поздних слоях для предсказания мэл-спектрограммы. На входе также действует слой с эмбеддингом входной последовательности и дополнительная 3x3 конволюция. На выходе стоит 1x1 конволюция с размером выхода в 80 -- длиной одного мэл вектора. Генератор мэл-спектрограмм был обучен с функций ошибки -- усредненной среднеквадратичной потерей (Mean-Square-Error, MSE) между элементами матриц.

Пустой $\sim$ символ дополняет алфавит исходной входной последовательности. Но вместо того, чтобы выделять для него отдельный вектор в таблице embeddings первого слоя генератора мэл-спектрограмм, он заменяется на линейную комбинацию эмбеддингов для соседним графем. Более точно, если пустой символ $\sim$ расположен между символами $a$ и $b$, его длительность равна $d$, то эмбеддинг $E$ для $\sim$ расположенного на расстоянии $t$ слева от $a$ будет равен $E (\sim, t) = \dfrac{d+1-t}{d+1} \cdot E(a) + \dfrac{t}{d+1} \cdot E (b)$. Это более точно соответствует смысловой нагрузке пустого символа, являющегося промежуточным символам в переходе от $a$ к $b$ и помогает модели быстрее обучаться.

\begin{table}[!ht]
\centering
\scalebox{1.2}{
\begin{tabular}{c c c c c} 
 \toprule
  \textbf{Block} &
  \textbf{\thead{\# Sub\\Blocks}} &
  \textbf{\thead{\# Output\\Channels}} &
  \textbf{Kernel Size} &
  \textbf{Dropout} \\
 \midrule
Embed & 1 & 256 & 1 & 0.0 \\
Conv1 & 3 & 256 & 3 & 0.0 \\
$B_1$ & 5 & 256 & 5 & 0.0 \\
$B_2$ & 5 & 256 & 7 & 0.0 \\
$B_3$ & 5 & 256 & 9 & 0.0 \\
$B_4$ & 5 & 256 & 13 & 0.0 \\
$B_5$ & 5 & 256 & 15 & 0.0 \\
$B_6$ & 5 & 256 & 17 & 0.0 \\
$B_7$ & 5 & 512 & 21 & 0.0 \\
$B_8$ & 5 & 512 & 23 & 0.0 \\
$B_9$ & 5 & 512 & 25 & 0.0 \\
Conv2 & 1 & 1024 & 1 & 0.0 \\
Conv3 & 1 & 80 & 1 & 0.0 \\
\midrule
\textbf{Params, M} & & & & \textbf{8.5} \\
\bottomrule
\end{tabular}
}
\caption{Параметры генератора мэл-спектрограмм с архитектурой, основанной на QuartzNet 9x5}
\label{tab:mels-model}
\end{table}

Глубина нейронной сети в 45 слоев (9 блоков по 5 подблоков) позволяет получить receptive field на последних слоях, накрывающий всю входную последовательность. Однако, он имеет неравномерную силу действия в зависимости от удаленности по времени. Это имеет простую интерпретацию: чем дальше от графемы (буквы) находиться другая графема, тем меньше она влияет на правильность произношения.

В общем и целом, представленная архитектура спроектирована таким образом, чтобы максимально устранить узкие места с точки зрения скорости и эффективности реализации на графических ускорителях. Отсутствие операций с механизмом внимания, широко использующимся в других подходов, позволяет получить максимальный эффект в рамках модели многопоточности GPU. Неавторегрессионность обоих шагов, а также явное предсказание длительностей, позволяет получить быструю устойчивую модель, сохраняя при этом возможность сохранить качество.  % 2
\section{Решение}

\subsection{Данные для обучения}

Стандартной практикой в области генерации данных является разделение датасетов на одноголосные (single-speaker) и многоголосные (multi-speaker), количество спикеров у которых доходит до нескольких тысяч, а количество минут чистой речи на каждого спикера -- около $20$~\cite{libritts}. Предварительные эскперименты с TalkNet показали, что для обучения эффективной многоголосной системы потребуется дополнительная контекстная информация о характере голоса, без которой сложно уловить зависимости между аудио и получить хорошее качество звучания. К примеру, в качестве эмбеддинга спикера можно использовать Global Style Token~\cite{wang2018style} или похожие эксперименты. Multi-Speaker TalkNet оставлено авторами как одно из направлений для будущей работы.

В рамках данной работы мы решили использовать данные из датасета LJSpeech~\cite{ljspeech}, который является стандартом де-факто для тестирования процесса генерации и позволяет легко сравнивать результаты с другими подходами. LJSpeech это одноголосый набор данных из 13100 отрывков семи аудиокниг документального жанра на английском языке. Размер отрывков варьируется от 1 до 10 секунд. Каждому отрывку в соответствие поставлена текстовая транскрипция с сохранением пунктуации и нормализацией чисел, денежных знаков и некоторых сокращение. Суммарная протяженность аудио -- около 24 часов.

Мы произвольно разделили набор данных на три части: $12,500$ для обучения, 300 для валидации и 300 для тестирования. Для обработки текста была использована стандартная токенизация с понижением регистра и использованием пунктуации. Таким образом мы полностью сохраняем эспрессивность текста и помогаем модели правильным образом выучить паузы, повышение тона, выделение голосом частей текста и другие особенности речи.

Как было уже сказано выше, мы не предсказывает "сырую" аудио дорожку напрямую, а используем промежуточное представление в виде мэл-спектрограмм. Такое компактное представление строится конструктивно и детерменированно из аудио с помощью оконного преобразования Фурье. Суть этой операции в последовательном применении преобразования Фурье к коротким кусочкам речевого сигнала, домноженным на некоторую оконную функцию. Результат применения оконного преобразования —- это матрица, где каждый столбец является спектром короткого участка исходного сигнала (Фигура~\ref{fig:mel-example}).

\begin{figure}[!ht]
\centering
\includegraphics[width=1.0\textwidth]{images/mel-example.png}
\caption{Примеры полученных мэл-спектрограмм с выровненным текстом}
\label{fig:mel-example}
\end{figure}

Для построения мэлов мы использование библиотеку \texttt{librosa}. Мы преобразуем аудиосигналы в мэл-спектрограммы с помощью кратковременного преобразования Фурье (Short-Time Fourier Transform, STFT), используя размер окна в 50 мс с шагом в 12.5 мс, окно вида "hann" и логарифмируем результат. Более подробные характеристики преобразования: $\texttt{win\_length}=1024, \texttt{hop\_length}=256, \texttt{n\_fft}=1024, \texttt{low\_freq}=0$ и $\texttt{high\_freq}=80$.

\subsection{Обучение предсказателя длительности графем}

Как уже было сказано выше, часть TalkNet'а ответственная за предсказание длительности графен обучается отдельно. На вход такой модели подается текстовое представление токенизированное посимвольно. Далее, между каждыми соседними символами вставляется пустой (blank) символ, означающий промежуточное состояние для перевода дыхания, кратковременной паузы и итд (Рисунок~\ref{fig:durs}). Итого, общая длина входа удваивается, как и длинна выхода. Сама же модель предсказания основанна на нейронной неавторегрессионной конволюционной архитектуре QuartzNet 5x5.

Нейронная модель для предсказания длительности графемы обучалась с помощью оптимизатора Adam с $\beta_1=0.9,\beta=0.999,\epsilon=10^{-8}$, $\texttt{weight\_decay}={10}^{-6}$ и $\texttt{gradient\_norm\_clipping}=1.0$. Для $\texttt{learning\_rate}$ мы использовали cosine decay policy начиная от $10^{-3}$ до $10^{-5}$ с $\texttt{warmup}=0.02$. Для обчения использовались различные конфигурации вычислительных мощностей с 1 и 8 графическими процессорами (GPU) V100 с 16 и 32 гигабайтами видеопамяти. Мы использование батч размера 256 для одной GPU в 16GB и увеличивали $\texttt{learning\_rate}$ пропорционально увеличению мощностей. Такая конфигурация гиперпараметров позволила получить сходимость всего лишь за 200 эпох, что занимало около $1.3$ часа чистого времени на одной GPU и около $11$ минут на восьми GPU. Мы использовали обучение со смешанной точностью (mixed-precision,~\cite{micikevicius}), так как эмпирически было выявлено что такой подход позволяет получить почти двоекратное ускорение с сохранением точности.

Результаты можно увидеть на Таблице~\ref{tab:durs-results}. Как можно заметить, нам удалось получить почти 70\% точности с простой моделью, которая содердит около 2.3 миллиона весов. Более того, около 93\% предсказаний находятся на абсолютном расстоянии не более чем в 1.

\subsection{Обучение генератора мэл-спектрограмм}

Генератор мэл-спектрограм производится из символьной входной последовательности после операции расширения (expansion), в результате которой мы выравниваем длинну текста и длинну мэла по времени (Рисунок~\ref{fig:arch}). Такое выравнивание соотносит буквы с произносимыми звуками и переходами между ними, облегчая процесс генерации. В качестве длительностей графем для тренировки мы используем полученные на этапе извлечения. Таким образом, обучение генератора не требует использования предобученного предиктора длительностей и может выполняться парралельно. Сама же модель генератора основанна на нейронной неавторегрессионной конволюционной архитектуре QuartzNet 9x5.

Как видно из Таблицы~\ref{tab:durs-results}, точность предсказателя длительностей составляет около $70\%$. В то же время, количество классов находящихся на абсолютном расстоянии не более 1 - около $92\%$. Для того чтобы уменьшить несоотвествие, которое возникает на этапе вывода (inference), были применены аугментации для истинных длительностей подающихся к обоим частям модели. Такие аугментации должны отвечать некоторых заданным критериям:
\begin{enumerate}
    \item Сохранять сумму длительностей всех букв неизменной, так как она напрямую зависимт от длины мэл-спектрограммы.
    \item Быть несмещенными относительно истинных длительностей.
    \item Сила изменений для каждого символа должна быть пропорицональна длительности. Таким образом, мы будем изменять символы с большими длительностями чаще.
    \item Сила аугментации должна быть контролируема с заданным параметром.
\end{enumerate}

Одной из аументаций, удолетворяющих всем вышеперечисленным условия, может являться "биномиальная встряска" (binominal shake). Суть заключается в "обмене" длительностями соседних символов $l$ и $r$ по биномиальному распределению с заданым параметром $p$ и $n=\min(d_l, d_r)$. Каждый символа обменивается длительностями с двумя соседями, а направление обмена выбирается случайным образом с вероятностью $p=0.5$ (Рисунок~\ref{fig:aug}).

\begin{figure}[!ht]
\centering
\includegraphics[width=0.49\textwidth]{images/aug/before.png}
\vrule
\includegraphics[width=0.49\textwidth]{images/aug/after.png}
\caption{Пример применения аугментация для длительностей графем для соседних символов. Слева - до, справа - после.}
\label{fig:aug}
\end{figure}

Опытным путем было выяснено, что аугментации помогают качеству звучания и уменьшают эффект переобучения (overfitting). Для тренировки генератора мэл-спектрограм мы применяли "биномиальную встряску" с $p=0.05$ для аугментации истинных длительностей.

Для тренировки генератора мэлов использовался тот же набор гиперпараметров, что и для предсказателя графем. Мы использование батч размера 64 для одной GPU в 16GB и увеличивали $\texttt{learning\_rate}$ пропорционально увеличению мощностей. Такая конфигурация позволила получить сходимость всего лишь за 200 эпох, что занимало около $8$ часа чистого времени на одной GPU и около $2$ часов на восьми GPU. Мы использовали обучение со смешанной точностью (mixed-precision,~\cite{micikevicius}), так как эмпирически было выявлено что такой подход позволяет получить почти двоекратное ускорение с сохранением качества.

Таким образом, процесс обучения обеих частей TalkNet'а на сервере DGX-1 может занимать всего порядка 2-ух часов. Это сравнимо меньше $2-3$ дней которые требуются модели Tacotron2~\cite{tacotron2}. Такая особенность связана прежде всего с эффектом неавторегрессионности, а также отсутствием операций основанных на механизмах внимания (attention), которые обычно занимают порядка $O(T^2)$ времени в зависимости от длины $T$.  % 3
\section{Результаты}

\subsection{Качество аудио}

Одна из самых больших проблем с разработкой систем для генерации речи это отсутствие быстро (программно) вычислимой метрики, которая хорошо коррелирует с качеством произношения. Зачастую, в процессе разработки, качество приходиться мерить "на слух", замедляя таким образом исследования и ухудшая реальную требуемую корреляцию с человекоподобной речью. Для того, чтобы оценить качество TalkNet, следуя другим работам~\cite{fastspeech, tacotron2}, было решено провести случайный слепые тесты с дискретным оцениванием и усреднить результаты для того, чтобы получить примерное представление о работоспособности получившейся модели.

Для оценки качества был проведен эксперимент MOS (mean opinion score) для сгенерированной речи с использованием Amazon Mechanical Turk~\cite{mturk}. Сравнивались четыре набора подходов к генерации для тестовой части датасета LJSpeech: 1) Истинные аудио с речью; 2) Истинные мэл-спектрограммы, преобразованная в речь с помощью WaveGlow; 3) Tacotron 2 + WaveGlow и 4) TalkNet + WaveGlow. Были использованы реализации NVIDIA для Tacotron 2 и WaveGlow. Тестировались $100$ аудио примеров, где каждый пример был оценен не менее $10$ раз $10$ различными людьми. Также, были использованы дополнительные фильтры для отсева людей без высшего образования или людей не знающих английский язык для повышения стабильности оценивания. Оценивающим предлагалось проверить работоспособность гарнитуры, несколько раз прослушать отрывок и выбрать наиболее подходящую оценку на вопрос "насколько представленный отрывок похож на человеческую речь?". Баллы варьировались от $1.0$ до $5.0$ с шагом $0.5$ (всего -- 9 ступеней). Как можно заметить, качество речи TalkNet сравнимо к Tacotron 2 (Таблицы~\ref{tab:mos}).

\begin{table}[!ht]
\centering
\scalebox{1.5}{
\begin{tabular}{l c} 
\toprule
\textbf{Model} &
\textbf{MOS} \\
\midrule
Ground truth speech & $4.31 \pm 0.05$ \\
Ground truth mel + WaveGlow & $4.04 \pm 0.05$ \\
Tacotron 2 + WaveGlow & $3.85 \pm 0.06$ \\
\midrule
TalkNet + WaveGlow & $3.74 \pm 0.07$ \\
\bottomrule
\end{tabular}
}
\caption{Усредненные оценки MOS (mean opinion score) с $95\%$ доверительным интервалом}
\label{tab:mos}
\end{table}

Одна из дополнительных характеристик генерации, помимо качества звучания, это способность модели не пропускать слова и правильно произносить имеющиеся даже при сложных входных строках. Такая характеристика называется устойчивостью (robustness). Следуя FastSpeech~\cite{fastspeech}, были проведены эксперименты для сравнения устойчивости различных моделей на $50$ особенно сложных для генерации входных предложениях~\ref{fig:hard-examples}, считая вручную количество пропущенных или повторенных слов. Данные предложения представляют особенную сложность, из-за отсутствия правдоподобного контекста, поэтому генеративной модели приходится выводить корректное произношение почти побуквенно. В результаты экспериментов было обнаружено, что подобно FastSpeech, TalkNet устраняет ошибки связанные с пропущенными или повторяющимися словами. Таким образом, неавторегрессионый подход очень устойчив и не имеет ошибок с отсутствующими или повторяющимися словам, что делает его более выгодным по сравнению с авторегрессионными моделями TTS, такими как Tacotron 2 или Transformer TTS.

\begin{figure}[!ht]
\centering
\includegraphics[width=1.0\textwidth]{images/snippets/hard-examples.png}
\caption{Первые 15 из 50 сложных примеров для проверки генерации TTS систем}
\label{fig:hard-examples}
\end{figure}

\subsection{Скорость генерации}

Процесс вывода (inference) TalkNet описан на Рисунке~\ref{fig:arch}. Сначала, вставляются пустые символы в входной текст между каждыми двумя соседними. Полученная последовательность пропускается через предиктор длительностей графем. Входные данные предиктора длительностей затем корректируются для символов с длительностью $0$. Так избегаются неправильные предсказания длительностей для редких символов (знаков препинания), что позволяет оставить их после операции расширения. Исправленная последовательность символов расширяется при повторении каждого символа в соответствии с предсказанной длительностью. Вторая часть модели генерирует мэл-спектрограмму из развернутой последовательности графем, равной ей по длине.

Такой процесс имеет несколько узких мест с точки зрения скорости:
\begin{itemize}
    \item Матричные операции, необходимые для вывода предиктора длительностей.
    \item Матричные операции, необходимые для вывода генератора мэл-спектрограмм.
    \item Передача и трансформация данных между частями.
\end{itemize}

Очевидно, что все части вывода по времени зависят от длины текста и соответствующего мэла. Поэтому, для правильного сравнивания TTS систем для генерации речи замерялась задержка (latency), необходимая всем частям системы суммарно на получение мэл-спектрограммы. Получившиеся задержки усреднялись ее по множеству примеров с различной длиной. Заметим также, что этап вокодинга не включен во время задержки, но все еще необходим для построения полной системы для генерации речи. Обычно, вокодинг занимает в десятки раз больше времени и весов, поэтому все еще является узких местом всей системы.

Были сравнены задержки вывода TalkNet с Tacotron 2 и FastSpeech. Использовались реализации FastSpeech от NVIDIA, так как исходная код был недоступен на момент оценки. Чтобы измерить задержку, были сгенерированны мэл-спектрограммы с размером батча, равным 1 и проводим усреднение по 2048 образцам из тестового набора данных LJSpeech. Средняя длина мэл-спектрограммы при таком подходе составляет $520$. Сравнивались задержки используя один и то же аппаратное оборудование (hardware) -- один графический процессор V100. Как можно видеть, скорость вывода TalkNet значительно быстрее, чем Tacotron 2 и FastSpeech (Таблица~\ref{tab:lats}). Если же проводить вывод по 4 или 8 примеров за раз, то скорость повышается линейно, достигая $1300$ RTF.

\begin{table}[!ht]
\centering
\scalebox{1.2}{
\begin{tabular}{l l l r} 
\toprule
\textbf{Model} & 
\textbf{\thead{\# Batch\\size}} &
\textbf{\thead{Inference\\Latency, s}} &
\textbf{RTF} \\
\midrule
Transformer TTS~\cite{transformer-tts} & 1 & $6.735 \pm 3.969$ & $1.48 \pm 0.87$ \\
Tacotron 2~\cite{tacotron2} & 1 & $0.817 \pm 1\cdot 10^{-2} $ & $7.56 \pm 0.01$ \\
FastSpeech~\cite{fastspeech} & 1 & $0.029 \pm 2 \cdot {10}^{-4}$  & $221.01 \pm 1.75$ \\
\midrule
TalkNet & 1 & $0.019 \pm 1 \cdot {10}^{-5}$ & $328.65 \pm 4.76$ \\
TalkNet & 4 & $0.023 \pm 5 \cdot {10}^{-5}$ & $1048.80 \pm 21.75$ \\
TalkNet & 8 & $0.037 \pm 4 \cdot {10}^{-4}$ & $1340.09 \pm 8.90$ \\
\bottomrule
\end{tabular}
}
\caption{Задержка вывода TalkNet для генерации мэл-спектрограммы (без вокодера). Задержка была измерена с размером батча $1$ с использованием графического процессора V100 и усреднена по 2048 примерам из набора данных LJSpeech. Приведены усредненное время задержки и фактор реального времени (RTF) с доверительным интервалом $95\%$. Фактор реального времени показывает сколько секунд речи можно сгенерировать за одну секунду вычислений.}
\label{tab:lats}
\end{table}

Поскольку TalkNet не использует операции с механизмами внимания (attention), задержка вывода практически не зависит от длины входного сигнала~\ref{fig:len-lat}. Операции c attention обычно реализуются с перемножением матриц, вытянутых по длине вывода. Поэтому, такая операция занимает порядка $O(T^2)$ операция, что приводит к линейному росту времени при парралелизации на графических ускорителях ГПУ.

\begin{figure}[!ht]
\centering
\includegraphics[width=0.8\textwidth]{images/len-lat.png}
\caption{Влияние неавторегрессионной архитектуры и отсутствия механизмов внимания}
\label{fig:len-lat}
\end{figure}

Стоит также отметить, что текущим самым узким местом с точки зрения скорости вывода TalkNet является непосредственные вычисления операций нейронной сети (конволюции, нелинейности, батч нормализация, дропаут). Однако, текущие реализации CUDA кернелов (специальных функций для вычислений операций на графических ускорителях) для depthwise separable конволюций, из которых почти полностью состоит TalkNet, написаны не самым оптимальным образом. Полагается, что написание оптимальных функций для 1d time и 1x1 pointwise сверток с использованием идей fusing'а (когда две последовательные операции сокращаются в одну) может значительно ускорить работу предложенной модели.  % 4
\section*{Заключение}

В этой статье мы представляем TalkNet, полностью свернутую нейронную систему синтеза речи. Модель состоит из двух сверточных сетей: предиктора длительности графемы и генератора мэл-спектрограмм. Эта модель не требует другой модели преобразования текста в речь в качестве преподавателя. Выравнивание графемы основной истины извлекается из выходных данных CTC предварительно обученной модели распознавания речи.
\sep
In this paper, we present TalkNet, a fully convolutional neural speech synthesis  system. The model is composed of two convolutional networks: a grapheme duration predictor and a mel-spectrogram generator. The model does not require another text-to-speech model as a teacher. The ground truth grapheme alignment is extracted from the CTC output of a pretrained speech recognition model.

The explicit duration predictor practically eliminates skipped or repeated words. TalkNet achieves a comparable level of speech quality to Tacotron 2 and FastSpeech. The model is very compact. It has only $10.8$M parameters, almost 3x less than similar neural TTS models: Tacotron-2 has 28.2M, and FastSpeech has 30.1M parameters. Training TalkNet takes only around $2$ hours on a server with 8 V100 GPUs. The parallel mel-spectrogram generation makes the inference significantly faster.

Современные генеративные модели все еще обладают рядом фундаментальных проблем, которые не позволяют считать задачу генерации решенной. Ценность этой работы заключается не только в предложенном и описанном подходе, решавшем поставленную задачу, но и в трудностях, возникших при реализации и тестировании, указывающих на глобальные проблемы и очерчивающих границы применимости того или иного метода или модели.

В рамках данной работы можно выделить следующие основные результаты:
\begin{itemize}
    \item Удалось проанализировать текущие подходы к генерации текста, подробно изучены принципы и особенности работы генеративных моделей с дискретными значениями, намечены основные сложности и проблемы.
    \item Придуманы и описаны метрики для комплексной оценки качества генерации.
    \item Придуманы и реализованы способы, позволяющие справляться с существующими проблемами. Основная цель - увеличение длины генерируемых сэмплов, была успешно решена.
    \item Итоговая модель получилась не только эффективной в терминах метрик, но и интерпретируемой и гибкой. Свойство интерпретируемости, основанное на механизме внимания, не только поможет в дальнейшем правильно анализировать влияние того или иного изменения на результат генерации, но и правильно подбирать параметры при текущей реализации на новых данных.
\end{itemize}

Представленный подход, несмотря на свою простоту, открывает сразу несколько направлений для дальнейшего изучения и доработки.

The model, training recipe, and audio samples will be open sourced as part of the NeMo toolkit~\cite{nemo}. The authors thank Jon Cohen, Vitaly Lavrukhin, Jason Li, Christopher Parisien, and Joao Felipe Santos for the helpful feedback and review.  % *

\bibliographystyle{ugost2008ls}
\bibliography{main.bib}

\end{document}
